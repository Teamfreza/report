\documentclass[12pt,a4paper]{article}
\usepackage{graphicx}
\title{HOW STUDENTS CAN BUY A KIKOMANDO WITHOUT BEING SEEN.}
\begin{document}
	NAME:LOKOSE FREZA
		\newline
	STUDENT NUMBER: 215021655
	\newline
	REGISTRATION NUMBER: 15/U/21287
	\newline
	\tableofcontents
	
	
	
	\section{Introduction}
	The term “kikomando” is a luganda word which refers to mixture of fried beans and sliced chapatti.
	It is a very common type of business that is carried out by the locals and other tribes within Makerere university territories.
	Most business men who deal in kikomando business are categorized mainly into two depending on whether they charge their customers for beans or not:-
	\begin{enumerate}
	\item 
		The first category refers to those who charge their customers for the beans and the minimum price for this category of kikomando dealers is 1200/= where the 1000/= is for two chapatti and 200/= for the beans.
    
    
		
    \item
    	The last category are those who do not charge for beans and their minimum price is 1000/= which is for the chapatti.
	    Most students prefer to eat kikomando because first all, it is very cheap and affordable, secondly, it’s nearer and accessible to students at any time they need it. 
	  
	 \end{enumerate}	
	 
	\section{Discussion}
	Though most students like eating kikomando, they do not want to be seen buying it.
	
	The female students do not want to be seen by the males’ and vice versa.
	The reasons why students hide while purchasing kikomando are
	\begin{enumerate}	
	\item
		They feel ashamed when buying it because they think kikomando is not for campusers.
		
	
		
	\item
	
		They male students think they might get rejected by a lady when proposing to her.
	

		
	\item
	

		The female students feel no man would propose to them once seen purchasing kikomando.
		
	
		
	\item
	
		Students feel that when they are seen buying kikomando, it shows that one is from a very humble background which most students do not want the public to know about them. 
	\end{enumerate}	

	
	Due to such reasons, students have devised a method of how they can buy the kikomando without most people noticing it.  
	Students make sure that before they commence the journey to go and buy the kikomando, they have data in their phones for social media so that they either go while facebooking or whatsapping.
	When they reach the place, the first thing they do is to look around to make sure that there is no one around who knows them, and then he or she uses sign language to communicate with the seller. For example let’s assume the students wants a kikomando of 2 chapatti, what he/she does is that he/she shows two fingers, a sign of cutting and then stands aside facebooking or whatsapping while waiting for it being prepared. When it’s done, the student gets it so fast that he/she disappears without looking back even if someone calls their name from behind.

	\section{Conclusion}
	I think this is one of the best methods that students have devised to buy a kikomando without being seen.
	
	\section{Recommendations}
	
	I would recommend that such a method should be used by all students who are kikomando consumers since it has worked and it will still hold so long as the students apply the principles behind.
\end{document}